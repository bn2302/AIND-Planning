\documentclass{article}

\usepackage[backend=biber]{biblatex}
\bibliography{references.bib}

\begin{document}

\title{Methods in planning  \\ A historic review}

\paragraph*{Introduction}

Interest in using intelligent algorithms for solving planning problems arose
with the advent of computers. Linear programming, which can also be seen as
planning under linear constraints, was used heavily during the Second World War
\cite{linear_prog}.
It allowed to develop optimal plans for resource allocation. Algorithms for
planning put this to the next level, leaving the linearity imposed by linear
programming and allowing for arbitrary logical constraints, infinite planning
horizons and probabilistic effects. In the following major research results in
planning achieved in the last decades will be reviewed.

\paragraph*{Direct state search}

Initial planning algorithms were state space search algorithms. These uniformed
methods directly expanded the state space 

They allowed
for expanding 
The first planners were direct search algorithms. Allowing for solving the problem at a whole, with the problem of poor scalability. A* star search was known since the, where the advent of forward search was for sure even older. But the limitations of these algorithms for even moderate sized problems, made them impractibable at that time. Also given the memory and computation restrictions imposed on systems in the 1960 thus more advanced 

\paragraph*{Planning systems}

First planning system was Strips providing, being remarkably close to more modern planning languages, it provided the frist planning language. Allowing this level of abstraction of problems on machines was a milestone at that time

Warplan was a milestone in the development of planners. Written in PROLOG, a logical programming language it was comparable short 100 lines of code compared to earlier planning
algorithms. Also, did it allow for interleaving, which was introduced by REF. 

Graphplan was one major improvment to previous planning systems. It allowed for the 

Satplan

\paragraph*{Binary decision diagrams}

Another field of planning is the usage of boolean algebra. Here, also integer programming can be used, prodiving excellent algorithms for complex and large state spaces, such as IBM's ILog solvers.

\paragraph*{Conclusion \& outlook}

Although 60 years old, planning is a very active field in artificial intelligence. The description of complex problems in very defined high level languaes, such as PDDL, allows to create very efficient algorithms to automatically find solutions in extremly large state spaces. It will be intersting to see, how planning will benifit, and also fuel methods such as deep neural nets and deep reinforcement learning.

\printbibliography
\end{document}
