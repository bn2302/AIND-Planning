\documentclass{article}
\usepackage[letterpaper, top=1in, bottom=1in, left=0.5in, right=0.5in]{geometry}

\usepackage[backend=biber]{biblatex}
\bibliography{references.bib}

\begin{document}

\title{Methods in planning  \\ A historic review}

\maketitle

\paragraph*{Introduction}

Interest in using intelligent algorithms for solving planning problems arose
with the advent of computers. Linear programming, which can also be seen as
planning under linear constraints, was used heavily during the Second World War
\cite{linear_prog}. It allowed to develop optimal plans for resource allocation.
But the short comings imposed by linear programming, only linear
constraints and continous variables, lead to the development of
algorithms that overcame these burdens. These algorithms put planning to the
next level. These algorithms allowed for logical constraints,
discontinous states spaces, infinite planning horizons and probabilistic
effects. Here, the major result in planning research are briefly review.

\paragraph*{Direct state search}

First planning algorithms were state space search algorithms. Breath first
search can be traced back to dynamic programming {Bellman}. Uniformed cost
search to the Dijkstra's shortest path algorithm. These algorithms, being
uniformed in nature, all suffered from the curse of dimensionality. Considering
the computers of the 1960's this was a huge burden for using those algorithms
at this time.

A* search, which used a heuristic function to approximate the distance to the
goal, proved very successful in overcoming the limitations of uniformed search.
But coming up with a good heuristics required human intelligence.
Thus the next improvement of artificial intelligence was the development of
algorithms that automatically device a heuristics.

\paragraph*{Planning systems}

First planning system was Strips providing, being remarkably close to more
modern planning languages, it provided the frist planning language. Allowing
this level of abstraction of problems on machines was a milestone at that time

Warplan was a milestone in the development of planners. Written in PROLOG, a
logical programming language it was comparable short 100 lines of code compared
to earlier planning algorithms. Also, did it allow for interleaving, which was
introduced by REF. 

Satplan

Graphplan was one major improvment to previous planning systems. It allowed for
the 

\paragraph*{Binary decision diagrams}

Another field of planning is the usage of boolean algebra. Here, also integer
programming can be used, providing excellent algorithms for complex and large
state spaces, such as IBM's ILog solvers.

\paragraph*{Conclusion \& outlook}

Although 60 years old, planning is a very active field in artificial
intelligence. The description of complex problems in very defined high level
languages, such as PDDL, allows to create very efficient algorithms to
automatically find solutions in extremely large state spaces. It will be
interesting to see, how planning will benefit, and also fuel methods such as
deep neural nets and deep reinforcement learning.

\printbibliography
\end{document}
