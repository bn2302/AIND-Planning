\documentclass{article}
\usepackage[letterpaper, top=1in, bottom=1in, left=0.5in, right=0.5in]{geometry}
\usepackage{listings}
\usepackage{booktabs}
\usepackage{caption}
\usepackage[backend=biber]{biblatex}
\bibliography{references.bib}

\begin{document}
\title{Project 3: Implement a planning search  \\ Research report}
\author{Batian Niebel}
\date{\today}
\maketitle

\paragraph*{Implementation}
The loading \& unloading actions, level-sum heuristics, and problem 1-3 were
implemented in \texttt{my\_air\_cargo\_problems.py} . The methods required to
build the planning graph algorithm were implemented in
\texttt{my\_planning\_graph.py}. The implementation was tested using the
provided unit tests.

The different algorithms together with problem 1-3 were compared using
\texttt{run\_analysis.py}. The results were stored in
\texttt{search\_result.json} and further processed with the ipython notebook
\texttt{analyze\_results.ipynb}.

\paragraph*{Analysis of the results}
All searches except the breadth first search yielded the same path length for
the individual problems (cf. table \ref{tab:overview}). The optimal plans for
the problems (cf listing \ref{lst:plan1}, \ref{lst:plan2} \& \ref{lst:plan3}) was taken from
the A* search with the planning graph and level-sum heuristics.

The planning graph implementation took exceptionally long, probably due to a not
optimized implementation and the small size of the problems. The overhead of the
planning graph does not out weight the uniformed searches. However the planning
graph shined as expected in the number of expanded nodes, thus if node
expansion gets more expensive (larger problems) it is for sure a feasible
option.

In general the A* search with the level sum heuristics performed best. The depth first search
found not the optimal solutions, due to its tendency to get stuck in local
minimum, when not further expanding certain trees, it is a poor choice for these
kind of planning problems \cite{russell_92}. The A* star heuristics doesn't have
this problem since the level sum heuristics is an admissable heuristics, i.e. it never
overestimates the cost, and the planning graph heuristics is also consistent,
therefore it fullfills the optimality condition \cite{russell_94}. Also the
breadth first and uniform cost search are optimal \cite{russell_92}.

\begin{center}
\captionof{table}{Overview of all results}\label{tab:overview}
\begin{footnotesize}
\begin{tabular}{lllllll}
\toprule
Algorithm & Air Cargo  & Expansions & Goal tests & New nodes & Plan length & Time elapsed \\
\midrule
astar\_search-h\_ignore\_preconditions & Problem 1 &         41 &         43 &       170 &           6 &    0.0523106 \\
                 & Problem 2 &       1450 &       1452 &     13303 &           9 &      5.33969 \\
                 & Problem 3 &       5040 &       5042 &     44763 &          12 &      20.5908 \\
astar\_search-h\_pg\_levelsum & Problem 1 &         11 &         13 &        50 &           6 &     0.673669 \\
                 & Problem 2 &         86 &         88 &       841 &           9 &      58.2366 \\
                 & Problem 3 &        365 &        367 &      3345 &          12 &      387.304 \\
breadth\_first\_search- & Problem 1 &         43 &         56 &       180 &           6 &    0.0424407 \\
                 & Problem 2 &       3346 &       4612 &     30534 &           9 &      15.8045 \\
                 & Problem 3 &      14120 &      17673 &    123927 &          12 &      110.124 \\
depth\_first\_graph\_search- & Problem 1 &         12 &         13 &        48 &          12 &     0.014013 \\
                 & Problem 2 &        107 &        108 &       959 &         105 &     0.398715 \\
                 & Problem 3 &       3752 &       3753 &     30138 &         293 &      17.9365 \\
uniform\_cost\_search- & Problem 1 &         55 &         57 &       224 &           6 &    0.0514757 \\
                 & Problem 2 &       4853 &       4855 &     44041 &           9 &      14.3754 \\
                 & Problem 3 &      18236 &      18238 &    158317 &          12 &      61.8446 \\
\bottomrule
\end{tabular}
\end{footnotesize}
\end{center}

\lstset{
  caption=Plan for Problem 1\label{lst:plan1}, 
  basicstyle=\footnotesize, frame=tb,
  xleftmargin=.2\textwidth, xrightmargin=.2\textwidth
  }

\begin{lstlisting}
Load(C1, P1, SFO)
Fly(P1, SFO, JFK)
Load(C2, P2, JFK)
Fly(P2, JFK, SFO)
Unload(C1, P1, JFK)
Unload(C2, P2, SFO)
\end{lstlisting}

\newpage

\lstset{
  caption=Plan for Problem 2 \label{lst:plan2}, 
  basicstyle=\footnotesize, frame=tb,
  xleftmargin=.2\textwidth, xrightmargin=.2\textwidth
}
\begin{lstlisting}
Load(C1, P1, SFO)
Fly(P1, SFO, JFK)
Load(C2, P2, JFK)
Fly(P2, JFK, SFO)
Load(C3, P3, ATL)
Fly(P3, ATL, SFO)
Unload(C3, P3, SFO)
Unload(C2, P2, SFO)
Unload(C1, P1, JFK) 
\end{lstlisting}

\lstset{
  caption=Plan for Problem 3\label{lst:plan3}, 
  basicstyle=\footnotesize, frame=tb,
  xleftmargin=.2\textwidth, xrightmargin=.2\textwidth
}
\begin{lstlisting}
Load(C2, P2, JFK)
Fly(P2, JFK, ORD)
Load(C4, P2, ORD)
Fly(P2, ORD, SFO)
Load(C1, P1, SFO)
Fly(P1, SFO, ATL)
Load(C3, P1, ATL)
Fly(P1, ATL, JFK)
Unload(C4, P2, SFO)
Unload(C3, P1, JFK)
Unload(C2, P2, SFO)
Unload(C1, P1, JFK) 
\end{lstlisting}

\printbibliography

\end{document}

