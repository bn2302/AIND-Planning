\documentclass{article}
\usepackage[letterpaper, top=1in, bottom=1in, left=0.5in, right=0.5in]{geometry}

\usepackage[backend=biber]{biblatex}
\bibliography{references.bib}

\begin{document}

\title{Methods in planning  \\ A historic review}
\author{Batian Niebel}
\date{\today}

\maketitle

\paragraph*{Introduction}

Interest in using intelligent algorithms for solving planning problems arose
with the advent of computers. Linear programming, which can also be seen as
planning under linear constraints, was used heavily during the Second World War
\cite{linearprog}. It allowed developing optimal plans for resource allocation.
But the shortcomings imposed by linear programming, only linear
constraints and continuous variables, lead to the development of
algorithms that overcame these burdens. These algorithms put planning to the
next level. These algorithms allowed for logical constraints,
discontinuous states spaces, infinite planning horizons and probabilistic
effects. Here, the major result of planning research are briefly reviewed.

\paragraph*{Direct state search}

First planning algorithms were state-space search algorithms. Breadth first
search can be traced back to dynamic programming \cite{bellman}. Uniformed cost
search to the Dijkstra's shortest path algorithm \cite{cormen}. These
algorithms, being uninformed in nature, all suffered from the curse of
dimensionality. Considering the computers of the 1960's this was a huge burden
for using those algorithms at this time.

A* search \cite{a_star}, which used a heuristic function to approximate the
distance to the goal, proved very successful in overcoming the limitations of 
uninformed search algorithms. But coming up with a good heuristics required
human intelligence. Thus the next improvement of artificial intelligence was the
development of algorithms that automatically device heuristics for the use in
A* search and similar algorithms.

\paragraph*{Planning systems}

In the 1970's, one of the first planning systems, \textsc{Strips}, introduced a
formal domain independent language for the description of planning problems
\cite{strips}. \textsc{Strips} was already remarkably close to 
\textsc{PDDL} \cite{pddl}, a planning problem
description language from the 1990's.
Theses planning languages provided the basis for algorithms that exploited the
formal abstraction for the design of automatic planning heuristics.  

\textsc{Satplan} transforms the planning problem into boolean satisfiability
problems \cite{satplan}, which are also commonly solved in circuit design or
automatic theorem proving. This satisfiability problem is then solved algorithms such as the
Davis-Putnam-Logemann-Loveland algorithm \cite{dpll}. \textsc{Graphplan}
iteratively expands the planning problem into a directed graph, where the nodes
are actions and atomic facts (states)\cite{graphplan}. By introducing
constraints which exclude mutually exclusive nodes, \textsc{Graphplan} expands
the state space into the direction of possible solutions, thus allowing for
planning in large state spaces.

\paragraph*{Other planners}

Another field of planning is the usage of boolean algebra. Here, also integer
programming can be used, providing excellent algorithms for complex and large
state spaces, such as IBM's ILog solvers \cite{ilog}. Markov decision processes
allow developing planning algorithms considering uncertainty \cite{mdp}. 

\paragraph*{Conclusion \& outlook}

Although 60 years old, planning is a very active field in artificial
intelligence. The description of complex problems in very defined high level
languages, such as \textsc{PDDL}, allows to create very efficient algorithms to
automatically find solutions in extremely large state spaces. It will be
interesting to see, how planning will benefit, and also fuel methods such as
deep neural nets and deep reinforcement learning in the near future.

\printbibliography
\end{document}
